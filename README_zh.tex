\PassOptionsToPackage{quiet}{fontspec}
\documentclass[UTF8]{ctexart}
\usepackage{amsmath}
\usepackage{amssymb}
\usepackage{graphicx}
\usepackage{booktabs}
\usepackage{hyperref}
\usepackage{geometry}
\usepackage{subfigure}
\usepackage{listings}
\usepackage{xcolor}

\geometry{a4paper, margin=1in}

\hypersetup{
    colorlinks=true,
    linkcolor=blue,
    filecolor=magenta,      
    urlcolor=cyan,
    pdftitle={基于深度估计和不确定性引导的3D目标检测研究分析},
    pdfpagemode=FullScreen,
}

% Configure listings to handle Chinese and better formatting
\lstset{
    basicstyle=\ttfamily\small,
    breaklines=true,
    frame=single,
    backgroundcolor=\color{gray!10},
    keywordstyle=\color{blue},
    commentstyle=\color{green!40!black},
    stringstyle=\color{red},
    extendedchars=false,
    keepspaces=true,
}

\title{基于深度估计和不确定性引导的3D目标检测研究分析}
\author{MonoDDLE Team}
\date{\today}

\begin{document}

\maketitle

\section{项目简介}

\textbf{MonoDDLE} (Monocular Dense Depth Distillation for Localization Errors) 是基于 MonoDLE 的改进版本。\textbf{论文暂未发表。}

3D目标检测的核心难点在于从单张RGB图像中恢复丢失的深度信息。现有的主流方法通常依赖稀疏的 LiDAR 点云真值进行监督训练,存在稀疏性和数据获取成本高的局限性。本项目旨在利用视觉基础模型(如 Depth Anything V3)的绝对度量深度作为“软标签”或“密集监督信号”,通过知识蒸馏的方式,指导轻量级单目检测器学习更鲁棒的深度特征,从而在不增加推理成本的前提下显著提升检测精度。

\begin{figure}[htbp]
    \centering
    \includegraphics[width=0.8\textwidth]{docs/images/combined3.png} \\
    \includegraphics[width=0.8\textwidth]{docs/images/combined2.png} \\
    \includegraphics[width=0.8\textwidth]{docs/images/combined1.png}
    \caption{项目可视化结果概览}
\end{figure}

\section{可视化结果}

KITTI 数据集上的部分可视化结果(图像编号:001230):

\begin{figure}[htbp]
    \centering
    \subfigure[2D 边界框]{
        \includegraphics[width=0.42\textwidth]{docs/images/2d.png}
    }
    \subfigure[3D 边界框]{
        \includegraphics[width=0.42\textwidth]{docs/images/3d.png}
    }
    \subfigure[DA3 深度伪标签]{
        \includegraphics[width=0.42\textwidth]{docs/images/da3.png}
    }
    \subfigure[深度不确定性]{
        \includegraphics[width=0.42\textwidth]{docs/images/unc.png}
    }
    \subfigure[原图]{
        \includegraphics[width=0.42\textwidth]{docs/images/img.png}
    }
    \subfigure[目标中心点热力图]{
        \includegraphics[width=0.42\textwidth]{docs/images/hm.png}
    }
    \subfigure[LiDAR BEV 投影]{
        \includegraphics[width=0.42\textwidth]{docs/images/lidar.png}
    }
    \caption{KITTI 数据集可视化展示}
\end{figure}

\section{实验结果}

我们在 KITTI 数据集上进行了广泛的实验,以下是部分核心实验结果:

\subsection{核心结果对比 (KITTI Validation Set)}

\begin{table}[htbp]
    \centering
    \small
    \resizebox{\textwidth}{!}{
    \begin{tabular}{lcccc}
        \toprule
        Method & 3D@0.7 (E/M/H) & BEV@0.7 (E/M/H) & 3D@0.5 (E/M/H) & BEV@0.5 (E/M/H) \\
        \midrule
        CenterNet & 0.60 / 0.66 / 0.77 & 3.46 / 3.31 / 3.21 & 20.00 / 17.50 / 15.57 & 34.36 / 27.91 / 24.65 \\
        MonoGRNet & 11.90 / 7.56 / 5.76 & 19.72 / 12.81 / 10.15 & 47.59 / 32.28 / 25.50 & 48.53 / 35.94 / 28.59 \\
        MonoDIS & 11.06 / 7.60 / 6.37 & 18.45 / 12.58 / 10.66 & --- & --- \\
        M3D-RPN & 14.53 / 11.07 / 8.65 & 20.85 / 15.62 / 11.88 & 48.53 / 35.94 / 28.59 & 53.35 / 39.60 / 31.76 \\
        MonoPair & 16.28 / 12.30 / 10.42 & 24.12 / 18.17 / 15.76 & 55.38 / 42.39 / 37.99 & 61.06 / 47.63 / 41.92 \\
        MonoDLE (Re-impl.) & 15.17 / 12.10 / 10.82 & 21.10 / 17.20 / 15.10 & 50.70 / 38.91 / 34.82 & 56.94 / 43.74 / 38.41 \\
        \textbf{MonoDDLE (Ours)} & \textbf{18.49 / 14.48 / 12.14} & \textbf{26.38 / 20.12 / 17.89} & \textbf{59.80 / 43.89 / 39.27} & \textbf{65.10 / 48.85 / 42.97} \\
        \bottomrule
    \end{tabular}
    }
    \caption{KITTI 验证集上的 AP 表现对比}
\end{table}

\subsection{消融实验:DA3 深度与不确定性}

\begin{table}[htbp]
    \centering
    \resizebox{\textwidth}{!}{
    \begin{tabular}{lcccc}
        \toprule
        Method & DA3 Depth & Uncertainty & 3D AP$_{R40}$ (E / M / H) & BEV AP$_{R40}$ (E / M / H) \\
        \midrule
        Baseline & & & 15.17 / 12.10 / 10.82 & 21.10 / 17.20 / 15.10 \\
        + DA3 & \checkmark{} & & 18.27 / 14.26 / 11.96 & 25.59 / 19.65 / 16.79 \\
        \textbf{+ Uncertainty} & \checkmark{} & \checkmark{} & \textbf{18.49 / 14.48 / 12.14} & \textbf{26.38 / 20.12 / 17.89} \\
        \bottomrule
    \end{tabular}
    }
    \caption{消融实验结果}
\end{table}

\subsection{模型参数量与计算量对比}

\begin{table}[htbp]
    \centering
    \resizebox{\textwidth}{!}{
    \begin{tabular}{llcc}
        \toprule
        Model & Backbone & FLOPs (G) & Params (M) \\
        \midrule
        MonoDLE & DLA-34 & 79.37 & 20.31 \\
        \textbf{MonoDDLE (Ours)} & \textbf{DLA-34} & \textbf{83.91} & \textbf{20.46} \\
        & HRNet-W32 & 212.25 & 48.91 \\
        & ResNet-50 & 439.70 & 91.41 \\
        & ConvNeXt-Tiny & 129.83 & 38.34 \\
        \bottomrule
    \end{tabular}
    }
    \caption{计算复杂度对比}
\end{table}

\section{使用说明}

\subsection{环境安装}

推荐使用 \texttt{uv} 管理环境:

\begin{lstlisting}[language=bash]
cd /path/to/MonoDDLE
uv venv .venv
source .venv/bin/activate
uv pip install torch torchvision --index-url https://download.pytorch.org/whl/cu128
uv pip install -r requirements.txt
\end{lstlisting}

\subsection{数据准备}

请下载 KITTI 数据集并按以下结构组织:

\begin{lstlisting}
MonoDDLE
|-- data
|   |-- KITTI
|       |-- ImageSets
|       |-- training
|       |   |-- calib
|       |   |-- image_2
|       |   |-- label_2
|       |-- DA3_depth_results
|           |-- 000000.npz
|           |-- ...
\end{lstlisting}

生成 DA3 深度数据的脚本:

\begin{lstlisting}[language=bash]
python tools/generate_da3_depth.py --data_path data/KITTI --split training
\end{lstlisting}

\subsection{训练与评估}

\paragraph{单进程 DP}
\begin{lstlisting}[language=bash]
python tools/train_val.py --config experiments/configs/monodle/kitti_da3_uncertainty.yaml
\end{lstlisting}

\paragraph{多进程 DDP}
\begin{lstlisting}[language=bash]
bash experiments/scripts/train_ddp.sh experiments/configs/monodle/kitti_da3_uncertainty.yaml
\end{lstlisting}

\paragraph{仅评估}
\begin{lstlisting}[language=bash]
python tools/train_val.py --config experiments/configs/monodle/kitti_da3_uncertainty.yaml -e
\end{lstlisting}

\section{DA3 深度蒸馏与不确定性}

\subsection{DA3 深度蒸馏}

本项目使用 Depth Anything V3 作为教师模型,预先生成全图稠密度量深度图作为伪标签,并通过蒸馏损失对检测器的深度预测头进行额外监督。

蒸馏总损失为:
\begin{equation}
L_{\text{total}} = L_{\text{cls}} + L_{\text{bbox}} + L_{\text{dim}} + \lambda \cdot L_{\text{distill}}
\end{equation}

\subsection{不确定性引导的自适应蒸馏}

进一步引入逐像素不确定性预测 $\sigma_i$,使模型对 DA3 伪标签的置信度进行自适应建模:
\begin{equation}
L_{\text{distill}}^{\text{unc}} = \frac{1}{N} \sum_{i} \frac{|d_i - \hat{d}_i|}{\sigma_i} + \log \sigma_i
\end{equation}

\section{致谢}

感谢 MonoDLE 和 CenterNet 的优秀开源实现。

\section{许可证}

本项目采用 MIT License 开源。

\end{document}
